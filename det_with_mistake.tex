\documentclass[12pt]{article}
%\documentclass[12pt]{revtex4}
%\documentclass[draft,12pt]{article}
\usepackage{amsmath}
\usepackage{amsfonts}
\usepackage{amsbsy}
\usepackage{lscape} 
\usepackage{color}
\usepackage{graphicx,epsfig}
\usepackage[english]{babel}
\usepackage{latexsym}
\usepackage{amssymb}
\usepackage{multirow}
%\usepackage{palatino}
%\usepackage{chancery}
%\usepackage{newcent}
%\usepackage{charter}
%\usepackage{zapfchan}
%\usepackage{bookman}
%\usepackage{sparticles}        %Package for displaying sparticle names. 
%\usepackage{feynmf}            %Package for feynman diagrams. 

\newcommand{\beq}{\begin{equation}}
\newcommand{\eeq}{\end{equation}}


\newcommand{\p}{\partial}
\newcommand{\mc}[1]{\mathcal{#1}}
\newcommand{\md}{\mathcal{D}}
\newcommand{\wt}{\widetilde}
\newcommand{\ov}{\overline}
\newcommand{\suc}{{\rm SU}_{\rm C}(3)}
\newcommand{\sul}{{\rm SU}_{\rm L}(2)}
\newcommand{\ue}{{\rm U}(1)}
\newcommand{\GeV}{{\rm GeV}}
\newcommand{\eV}{{\rm eV}}
%\newcommand{\su3}{{\rm SU}_{\rm C}(3)}
%%%%%%%%%%%%%%%%%%%%%%%%%%%%%%%%%%%%%%%
%  Slash character...
\def\slashed#1{\setbox0=\hbox{$#1$}             % set a box for #1
   \dimen0=\wd0                                 % and get its size
   \setbox1=\hbox{/} \dimen1=\wd1               % get size of /
   \ifdim\dimen0>\dimen1                        % #1 is bigger
      \rlap{\hbox to \dimen0{\hfil/\hfil}}      % so center / in box
      #1                                        % and print #1
   \else                                        % / is bigger
      \rlap{\hbox to \dimen1{\hfil$#1$\hfil}}   % so center #1
      /                                         % and print /
   \fi}                                        %

%%EXAMPLE:  $\slashed{E}$ or $\slashed{E}_{t}$

\newcommand{\x}{x_0^\mu}



\begin{document}

    Let the ``super''-center of the string be defined by three functions: $ x_0^\mu(\zeta^a) $, $ \epsilon(\zeta^a) $
and $\ov\epsilon(\zeta^a)$, where $ \zeta^a $ are the worldsheet coordinates, and in essense are just $ x^0 $ and $ x^3 $.

    As we put the string into the position $ x_0^\mu $, we introduce a shifted reference frame. At each point of the string
we determine two unit normal vectors to the string, $ n^\mu_A $, and introduce the coordinates in the orthogonal plane defined
by these two vectors --- $ \rho^A $.
An arbitrary point in the superspace is referenced as
\[
	\Upsilon_M ~~=~~ \left\{ x_0^\mu ~+~ \rho^A n^\mu_A~,~~ \theta' ~+~ \epsilon~,~~ \ov{\theta}' ~+~ \ov{\epsilon} \right \}~,
\]
here $ \theta' $ and $ \ov{\theta}' $ are new supercoordinates.
In fact, in terms of the new coordinates the expression above does not look complete, as we have essentially done a shift
\[
	\theta ~~\to~~ \theta ~+~ \epsilon ~, \qquad\qquad \ov{\theta} ~~\to~~ \ov{\theta} ~+~ \ov{\epsilon}~,
\]
which is a supertransformation, and $ x_0^\mu $ has to be modified in accord:
\[
    x_0^\mu ~~\to~~ x_0^\mu ~+~ i\, \epsilon\slashed{\p}\ov{\theta}'(x_0^\mu) ~+~ i\, \ov{\epsilon\slashed{\p}}\theta'(x_0^\mu)~.
\]
Then $ \Upsilon_M $ looks as 
\[
	\Upsilon_M ~~=~~ \left\{ x_0^\mu ~+~ \rho^A n^\mu_A ~+~ i\, \epsilon\slashed{\p}\ov{\theta}' x_0^\mu ~+~ 
								i\, \ov{\epsilon\slashed{\p}}\theta' x_0^\mu~,
	~~ \theta' ~+~ \epsilon~,~~ \ov{\theta}' ~+~ \ov{\epsilon} \right \}~.
\]
We ignored here the small dependence of $ n_A^\mu $ along the string, the same way as it is done in the pure-bosonic case.

The new coordinates to which we pass are
\[
	\Xi^A =  \left\{  \zeta^a ~,~~ \rho^A ~,~~ \theta' ~,~~ \ov{\theta}' \right \}~.
\]

When calculating the action via
\[
	S ~~=~~ \int d^4\theta d^4x S(x,\theta,\ov{\theta})
\]
when we pass to the new coordinates, we have to calculate the Jacobian
\[
	\det \left\lgroup \frac{\p \Upsilon^M}{\p\Xi^A} \right\rgroup ~~=~~
		\sqrt{ \det \left\lgroup \frac {\p\Upsilon_M}{\p\Xi^A} \frac{\p\Upsilon^M}{\p\Xi^B} \right\rgroup }
\]


\newpage
The above matrix in the radical looks as follows
(we drop the primes on $ \theta' $ from now on).

\begin{flushleft}
\begin{tabular}{c||p{3.0cm}|p{3.0cm}|p{3.0cm}|p{3.0cm}}
% header line
	&   \parbox{3.0cm}{\centering $\zeta^b$}
	&   \parbox{3.0cm}{\centering  $\rho^B$}   
	&   \parbox{3.0cm}{\centering $\theta_\beta$}
	&   \parbox{3.0cm}{\centering $\ov{\theta}^{\dot{\beta}}$}   \\[1.0mm]
\hline
\hline
% first line
	\multirow{3}{*}
	{ $\zeta^a$ }  
&
	\multirow{3}{*}{    
	\parbox{3.0cm}{\centering
	$\p_a\x \p_b\x $ \\[1mm]
	$ ~+~ \p_a\epsilon\p_b\epsilon$ \\[1mm]
	$ ~+~ \p_a\ov{\epsilon}\p_b\ov{\epsilon}$ 
	}}
&
	\multirow{3}{*}{
	\parbox{3.0cm}{\centering
	---
	}}
&
	\multirow{3}{*}{
	\parbox{3.0cm}{\centering
		$i\,\p_a(\x)\ov{\epsilon^{\dot{\beta}}\slashed{\p}}_{\dot{\beta}}^{\ \beta}(\x) $ \\[1mm]
							$ ~-~ \p_a \epsilon^\beta$ }}
&   
	\multirow{3}{*}{
	\parbox{3.0cm}{\centering$i\,\p_a(\x)\epsilon_\beta\slashed{\p}^\beta_{\ \dot{\beta}}(\x) $ \\[1mm]
							$~-~ \p_a\ov{\epsilon}_{\dot{\beta}} $}}   \\[2mm]
&&&&\\
&&&&\\
% second line
\hline
	\multirow{3}{*}{
	\centering $\rho^A $
	}
&
	\multirow{3}{*}{
	\parbox{3.0cm}{\centering
	---
	}}
&
	\multirow{3}{*}{
	\parbox{3.0cm}{\centering $\delta^{AB}$
	}}
&
	\multirow{3}{*}{
	\parbox{3.0cm}{\centering
	---
	}}
&
	\multirow{3}{*}{
	\parbox{3.0cm}{\centering
	---
	}}
	\\
&&&&\\
&&&&\\
% third line
\hline
	\multirow{3}{*} { $\theta_\alpha$ } &
	\multirow{3}{*}{
	\parbox{3.0cm}{\centering
	$i\,\p_b(\x)\ov{\epsilon^{\dot{\beta}}\slashed{\p}}_{\dot{\beta}}^{\ \alpha}(\x)$ \\[1.0mm]
	$~-~ \p_b \epsilon^\alpha $
	}}
&
	\multirow{3}{*}{
	\parbox{3.0cm}{\centering
	---
	}}
&
	\multirow{3}{*}{
	\parbox{3.0cm}{\centering
	$\epsilon^{\alpha\beta} ~+~ \frac{1}{2} \epsilon^{\alpha\beta} \ov{\epsilon\epsilon} \cdot $  \\[1.0mm]
	$\p_a\x\p_a\x $
	}}
&
	\multirow{3}{*}{
	\parbox{3.0cm}{\centering 
	$-\, \ov{\epsilon^{\dot{\gamma}}\slashed{\p}}_{\dot{\gamma}}^{\ \alpha}(\x) \cdot $ \\[1.0mm]
	$ \epsilon_\beta \slashed{\p}^{\beta}_{\ \dot{\beta}}(\x) $
	}}\\ [2.0mm]
&&&&\\
&&&&\\
% fourth line
\hline
	\multirow{3}{*}{
	\centering
	$\ov{\theta}^{\dot{\alpha}}$
	}
&
	\multirow{3}{*}{
	\parbox{3.0cm}{\centering
	$ i\,\p_b(\x)\epsilon_\beta\slashed{\p}^\beta_{\ \dot{\alpha}}(\x) $ \\[1.0mm]
	$ -~ \p_b \ov{\epsilon}_{\dot{\alpha}} $
	}}
&
	\multirow{3}{*}{
	\parbox{3.0cm}{\centering
	---
	}}
&
	\multirow{3}{*}{
	\parbox{3.0cm}{\centering
	$-\, \epsilon_\gamma\slashed{\p}^{\gamma}_{\ \dot{\alpha}}(\x)\cdot$ \\[1.0mm]
	$\ov{\epsilon^{\dot{\beta}}\slashed{\p}}_{\dot{\beta}}^{\ \beta}(\x)$
	}}
&
	\multirow{3}{*}{
	\parbox{3.0cm}{\centering
	$\epsilon_{\dot{\alpha}\dot{\beta}} ~+~ 
	\frac{1}{2} \epsilon_{\dot{\alpha}\dot{\beta}} \epsilon\epsilon \cdot $ \\[1.0mm]
	$\p_a\x\p_a\x $
	}}\\ [2.0mm]
&&&&\\
&&&&\\
\end{tabular}
\end{flushleft}

	At this point we can fix the worldsheet coordinates (such as $ \zeta^a = x_0^0\,,~ x_0^3 $).

\end{document}
